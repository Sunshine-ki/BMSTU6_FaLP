\documentclass[a4paper,oneside,12pt]{extreport}

\include{preamble}


\begin{document}

\include{title}

\section*{Практическая часть}

\begin{task}
	Написать функцию, которая принимает целое число и возвращает первое чётное число, не меньшее аргумента.

	\begin{lstlisting}[language=Lisp]
(defun f1 (a) (cond  
	((= (rem a 2) 0) a) 
	((+ a 1)) ))
	\end{lstlisting}

	Результат работы:
	\begin{lstlisting}[language=Lisp]
(f1 11) ;;  12
(f1 10) ;;  10
	\end{lstlisting}
\end{task}


\begin{task}
	Написать функцию, которая принимает число и возвращает число того же знака, но с модулем на 1 больше модуля аргумента.

	\begin{lstlisting}[language=Lisp]
(defun f2 (a) (cond 
		((< a 0) (- a 1)) 
		(t (+ 1 a)) ))
	\end{lstlisting}

	Результат работы:
	\begin{lstlisting}[language=Lisp]
(f2 3)  ;;  4
(f2 -3) ;; -4
	\end{lstlisting}
\end{task}

\begin{task}
	Написать функцию, которая принимает два числа и возвращает список из этих чисел, расположенный по возрастанию

	\begin{lstlisting}[language=Lisp]
(defun f3 (a b) (cond 
		((> a b) (list b a))
		(T (list a b)) ))
	\end{lstlisting}

	Результат работы:
	\begin{lstlisting}[language=Lisp]
(f3 1 2) ;; (1 2)
(f3 2 1) ;; (1 2)
	\end{lstlisting}
\end{task}

\begin{task}
	Написать функцию, которая принимает три числа и возвращает T только тогда, когда первое число расположено между вторым и третьим.

	\begin{lstlisting}[language=Lisp]
(defun f4 (a b c) (cond ((and (> a b) (< a c)) T)
	((and (< a b) (> a c)) T)))
	\end{lstlisting}

	Результат работы:
	\begin{lstlisting}[language=Lisp]
(f4 2 1 3) ;;  T
(f4 2 3 1) ;;  T
(f4 1 2 3) ;;  Nil	
	\end{lstlisting}
\end{task}

\begin{task}
	Каков результат вычисления следующих выражений?

	При вычислении приведенного ниже выражения 
	последовательно слева направо вычисляются аргументы функции до тех пор, пока не
	встретится значение аргумента, равное NIL. 
	В этом случае вычисление прерывается и значение функции равно NIL. 
	Если же были вычислены все значения аргеументов и
	они все отличны от NIL, то результирующим значением
	функции and будет значение последнего аргумента.
	\begin{lstlisting}[language=Lisp]
(and 'free 'fir 'foe) ;; foe 
	\end{lstlisting}

	При вычислении приведенного ниже выражения 
	последовательно слева направо вычисляются аргументы
	до тех пор, пока не встретится значение аргумента, отличное от NIL.
	В этом случае вычисление прерывается и значение функции равно
	значению этого аргумента.
	Если же вычисленые значения всех аргументов равны NIL, 
	то результирующее значение функции
	равно NIL.
	\begin{lstlisting}[language=Lisp]
(or 'fee 'fie 'foe) ;; fee
	\end{lstlisting}

	Сначала вычисляется (equal 'abc 'abc) и так как вычисленное значение равно T
	продолжается вычисление аргументов. 
	'yes - вычисляется как T и является последним аргументом, поэтому он
	является результатом выполнения функции.
	\begin{lstlisting}[language=Lisp]
(and (equal 'abc 'abc) 'yes) ;; yes
	\end{lstlisting}

	Возвращает первый аргумент, отличный от nil, в нашем случае это второй аргумент.
	\begin{lstlisting}[language=Lisp]
(or nil 'fie 'foe) ;; fie
	\end{lstlisting}

	Возвращает nil, т.к. один из аргументов (в нашем случае первый) вычисляется как nil,
	\begin{lstlisting}[language=Lisp]
(and nil 'fie 'foe) ;; nil
	\end{lstlisting}

	Возвращает T, т.к. or - возвращает значение первого аргумента, отличного от нуля.
	В нашем случае это первый аргумент - его значение T.
	\begin{lstlisting}[language=Lisp]
(or (equal 'abc 'abc) 'yes) ;; T
	\end{lstlisting}

\end{task}

\begin{task}
	Написать предикат, который принимает два числа-аргумента и возвращает T, если первое число не меньше второго.

	\begin{lstlisting}[language=Lisp]
(defun f6 (a b) (cond ((>= a b) T)))
	\end{lstlisting}


	Результат работы:
	\begin{lstlisting}[language=Lisp]
(f6 2 1)	;;  T
(f6 5 5) 	;;  T
(f6 1 2) 	;;  Nil
	\end{lstlisting}
\end{task}

\begin{task}
	Какой из следующих двух вариантов предикатов ошибочен и почему?

	Предикат \code{numberp} возвращает	 T, если значение его аргумента - числовой атом, и NIL в противном случае.
	\code{plusp} возвращает T, если аргумент больше нуля, и NIL в противном случае.

	Приведенный ниже предикат возвращает T, если аргументом является числовой атом и он больше нуля.
	\begin{lstlisting}[language=Lisp]
(defun pred1 (x)
		(and (numberp x) (plusp x)) )
	\end{lstlisting}

	Результат работы:

	\begin{lstlisting}[language=Lisp]
(pred1 1)    ;; T
(pred1 0)    ;; Nil
(pred1 -1)   ;; Nil
(pred1 'a)   ;; Nil		
	\end{lstlisting}

	Приведенный ниже предикат ошибочен, потому что при вызове функции and
	аргументы вычисляются слева направо. При попытке проверить, является ли аргумент
	больше нуля (с помощью предиката plusp) возникает ошибка, если его аргументом не
	является числовой атом. 
	Поэтому изначально следует проверить, что аргумент является числовым атомом и только после 
	этого вызывать функцию plusp. %, которая принимает числовой атом.
	\begin{lstlisting}[language=Lisp]
(defun pred2 (x)
		(and (plusp x) (numberp x)) )
	\end{lstlisting}

	Результат работы:

	\begin{lstlisting}[language=Lisp]
(pred2 1)    ;; T
(pred2 0)    ;; Nil
(pred2 -1)   ;; Nil
(pred2 'a)   ;; Error:   The value A is not of type NUMBER
	\end{lstlisting}


\end{task}

\begin{task}
	Решить задачу 4, используя для её решения конструкции \code{IF}, \code{COND}, \code{AND}/\code{OR}.

	AND можно представить как:
	\begin{lstlisting}[language=Lisp]
	(cond (x y))
	\end{lstlisting}

	OR можно представить как:
	\begin{lstlisting}[language=Lisp]
	(cond (x) (y))
	\end{lstlisting}
	
	Используя \code{COND}.	
	\begin{lstlisting}[language=Lisp]
(defun f4 (a b c) (cond 
		((> a b) (< a c)) ((< a b) (> a c))) )
	\end{lstlisting}

	Используя \code{AND}/\code{OR}.	
	\begin{lstlisting}[language=Lisp]
(defun f4 (a b c) (or (and (> a b) (< a c)) (and (< a b) (> a c))) )
	\end{lstlisting}

	Используя  \code{IF}.	
	\begin{lstlisting}[language=Lisp]
(defun f4 (a b c) 
	(if (> a b) 
		(if (< a c) T Nil) (if (> a c) T Nil)) )
	\end{lstlisting}

\end{task}

\begin{task}
	Переписать функцию how-alike, приведенную в лекции и использующую \code{COND}, используя  конструкции \code{IF}, \code{AND}/\code{OR}.

	\begin{lstlisting}[language=Lisp]
(defun how-alike (x y)
	(cond ((or (= x y) (equal x y)) 'the_same)
		((and (oddp x) (oddp y)) 'both_odd)
		((and (evenp x) (evenp y)) 'both_even) 
		(t 'diff)) )
	\end{lstlisting}


	Используя  \code{IF}.	
	\begin{lstlisting}[language=Lisp]
(defun how-alike (x y)
	(if (or (= x y) (equal x y)) 'the_same
	(if (and (oddp x) (oddp y)) 'both_odd 
	(if (and (evenp x) (evenp y)) 'both_even 'diff))) )
	\end{lstlisting}


	Используя  \code{AND}/\code{OR}.	
	\begin{lstlisting}[language=Lisp]
(defun how-alike (x y)
	(or (and (or (= x y) (equal x y)) 'the_same)
		(and (and (oddp x) (oddp y)) 'both_odd)
		(and (and (evenp x) (evenp y)) 'both_even)
		'diff) )
	\end{lstlisting}

	Результат работы:

	\begin{lstlisting}[language=Lisp]
(how-alike 1 1)   ;; THE_SAME
(how-alike 2 4)   ;; BOTH_EVEN
(how-alike 7 11)  ;; BOTH_ODD
(how-alike 12 7)  ;; DIFF
	\end{lstlisting}

\end{task}

\newpage

\section*{Теоретическая часть}

\subsection*{Классификация функций}

\begin{enumerate}
	\item Чистые математические функции - имеет фиксированное количество аргументов и один результат.
	\item Специальны функции (формы) - произвольное количество аргументов.
	\item Псевдофункции - создают эффект на экране.
	\item Рекурсивные функции.
	\item Функции с вариантными значениями, которые возвращают одно значение.
	\item Функции высших порядков - используются для построения синтаксически управляемых программ. %(абстракции языка)
\end{enumerate}

\subsection*{Работа функций and, or, if, cond}


\begin{itemize}
	\item При работе функции \code{and}
	последовательно слева направо вычисляются аргументы функции до тех пор, пока не
	встретится значение аргумента, равное NIL. 
	В этом случае вычисление прерывается и значение функции равно NIL. 
	Если же были вычислены все значения аргеументов и
	они все отличны от NIL, то результирующим значением
	функции and будет значение последнего аргумента.
	\begin{lstlisting}[language=Lisp]
(and arg1 arg2 ... argn)
	\end{lstlisting}
	Пример:
	\begin{lstlisting}[language=Lisp]
(and 'one 'two 'three) ;; three 
	\end{lstlisting}
	
	\item При работе функции \code{or}
	последовательно слева направо вычисляются аргументы
	до тех пор, пока не встретится значение аргумента, отличное от NIL.
	В этом случае вычисление прерывается и значение функции равно
	значению этого аргумента.
	Если же вычисленые значения всех аргументов равны NIL, 
	то результирующее значение функции
	равно NIL.
	\begin{lstlisting}[language=Lisp]
(or arg1 arg2 ... argn)
	\end{lstlisting}

	Пример:
	\begin{lstlisting}[language=Lisp]
(or 'one 'two 'three) ;; one 
	\end{lstlisting}
	
	\item if.  В случае, если условие test истинно
	возвращается t\_body, иначе, если test ложно
	возвращается f\_body
	
	\begin{lstlisting}[language=Lisp]
(if test t_body f_body)
	\end{lstlisting}
	

	\item cond (сокращение от condition - условие) служит средством
	разветвления вычислений. 
	\begin{lstlisting}[language=Lisp]
(cond (test1 value1)
      (test2 value2)
	    ...
      (testn valuen) )
\end{lstlisting}

	Функция cond последовательно слева направо вычисляет testi (i=0...n) 
	до тех пор, пока что не встретит значение, отличное от Nil. 
	Результатом будет вычисленное значение valuei.  


\end{itemize}







\subsection*{Способы определения функции}


1. С помощью \code{lambda}. 

Базовое определение лямбда-выражения:

(lambda лямбда-список тело\_функции)

Пример:

\begin{lstlisting}[language=Lisp]
(lambda (x y) (+ x y))
\end{lstlisting}

Применение лямбда-выражений:

(лямбда-выражение фактические\_параметры)

Пример:

\begin{lstlisting}[language=Lisp]
((lambda (x y) (+ x y)) 1 2) ;; 3
\end{lstlisting}


2. С помощью \code{defun}. 
Для неоднократного применения функции (а также для построения
рекурсивной функции) используется встроенная функция defun.

Синтаксис:

(defun имя\_функции лямбда-список тело\_функции)

Пример:

\begin{lstlisting}[language=Lisp]
(defun sum (x y) (+ x y))
\end{lstlisting}

Пример вызова:

\begin{lstlisting}[language=Lisp]
(sum 1 2) ;; 3
\end{lstlisting}


\end{document}