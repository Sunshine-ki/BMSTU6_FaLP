\documentclass[a4paper,oneside,12pt]{extreport}

\include{preamble}


\begin{document}

\include{title}

\section*{Практическая часть}

\begin{task}
    Написать функцию, которая выбирает из заданного списка только
    те числа, которые расположены между двумя заданными границами.

    \begin{lstlisting}[language=Lisp]
(defun check-border (x a b)
    (and (>= x a) (<= x b)) )

(defun select-between (lst a b)
    (cond ((null lst) ())
    ((symbolp (car lst)) (cons (car lst) (select-between (cdr lst) a b)))
    ((listp (car lst)) (cons (select-between (car lst) a b) (select-between (cdr lst) a b)))
    ((check-border (car lst) a b) (cons (car lst) (select-between (cdr lst) a b)))
    (T (select-between (cdr lst) a b))) )
    \end{lstlisting}

    Пример использования:
    \begin{lstlisting}[language=Lisp] 
(select-between '(1 2 (a b 3 4) T c 4 6 11 5) 2 7) ;; (2 (A B 3 4) T C 4 6 5)
    \end{lstlisting}
\end{task}

\begin{task}
	Написать функцию, вычисляющую декартово произведение двух своих списков-аргументов.
	(Напомним, что $A\times B$ — это множество всевозможных пар $(a\;b)$, где $a\in A$, $b\in B$)

    \begin{lstlisting}[language=Lisp]
(defun decart-elem (lst elem)
	(cond ((null lst) ())
	(T (cons (list elem (car lst)) (decart-elem (cdr lst) elem)))) )
    \end{lstlisting}


    Пример использования:
    \begin{lstlisting}[language=Lisp]    
(decart-elem '(a b c) 'd) ;; => ((D A) (D B) (D C))
    \end{lstlisting}

    \begin{lstlisting}[language=Lisp]
(defun decart (lst1 lst2)
	(cond ((null lst1) nil)
	(T (append (decart-elem lst2 (car lst1)) (decart (cdr lst1) lst2)))) )
    \end{lstlisting}

    Пример использования:
    \begin{lstlisting}[language=Lisp]   
;; ((A D) (A E) (A F) (B D) (B E) (B F) (C D) (C E) (C F))
(decart '(a b c) '(d e f))  
    \end{lstlisting}
\end{task}

\begin{task}
    Почему так реализовано \code{reduce}, в чем причина? 

    \begin{lstlisting}[language=Lisp]
        (reduce #'+ ()) ; 0
        (reduce #'* ()) ; 1
    \end{lstlisting}
        
    $ L = (e_1 e_2 ... e_n )$

    (reduce F L) $\equiv$ (F ( ... (F (F $e_1$ initial-value) $e_2$)) ... $e_n$)

    \begin{lstlisting}[language=Lisp]
(reduce '+ '(1 2 3 4)) ;; 10
(reduce '* '(1 2 3 4)) ;; 24
    \end{lstlisting}

    Причина состоит в том, что если начальное значение при
    умножении будет равно нулю, то и результат будет равен нулю
    \begin{lstlisting}[language=Lisp]
(reduce '* '(1 2 3 4) :initial-value 0) ;; 0
    \end{lstlisting}

    В случае сложения результат будет на единицу больше, 
    что является некорректным результатом.
    \begin{lstlisting}[language=Lisp]
(reduce '+ '(1 2 3 4) :initial-value 1) ;; 11
    \end{lstlisting}
        

\end{task}

\begin{task}
	Пусть \code{list-of-list} список, состоящий из списков.
	Написать функцию, которая вычисляет сумму длин всех элементов \code{list-of-list}, т.~е. например для аргумента \code{((1 2) (3 4)) -> 4}.

    \begin{lstlisting}[language=Lisp]
(defun list-of-list-rec (lst len)
	(cond ((null lst) len)
	(T (list-of-list-rec (cdr lst) (+ len (length (car lst)) )))))

(defun list-of-list (lst)
	(list-of-list-rec lst 0))
    \end{lstlisting}

    Пример использования:
    \begin{lstlisting}[language=Lisp]    
(list-of-list '((1 2 3) (4 5))) ;; => 5
    \end{lstlisting}

    Вариант для структурированного списка:

    Примечание: +1 потому что нужно ещё учесть сам список (который дальше будет просматриваться).
    \begin{lstlisting}[language=Lisp]
(defun list-of-list-rec (lst len)
    (cond ((null lst) len)
        ((listp (car lst)) 
        (+ 1 (list-of-list-rec (car lst) 0) (list-of-list-rec (cdr lst) len))) 
        (T (list-of-list-rec (cdr lst) (+ 1 len))) ))

(defun list-of-list (lst)
    (- (list-of-list-rec lst 0) (length lst)))
     \end{lstlisting}
        
    Пример использования:
    \begin{lstlisting}[language=Lisp]  
(list-of-list '((1 2) (3 4) (5 6 (7 8) 9))) ;; => 10
    \end{lstlisting}
\end{task}

\begin{task}
    Написать функцию, возвращающую 
    квадрат смешанного стуктурированного списка.
    В результирующем списке только числа.
    \begin{lstlisting}[language=Lisp]
(defun square-lst (lst) 
	(cond ((null lst) Nil)
	((symbolp (car lst)) (square-lst (cdr lst)))
	((listp (car lst)) (append (square-lst (car lst)) (square-lst (cdr lst))))
	(T (cons (* (car lst) (car lst)) (square-lst (cdr lst)))) )) 
    \end{lstlisting}

    Пример использования:
    \begin{lstlisting}[language=Lisp]    
(square-lst '((1 2 a) 'b 3 T 4))    ;; => (1 4 9 16)
(square-lst '((((1 2 3)))))         ;; => (1 4 9)
    \end{lstlisting}
\end{task}


% \begin{figure}[ht!]
% 	\centering{
% 		\includegraphics[width=0.5\textwidth]{img/5.png}
% 		\caption{Результат работы 5.} }
% \end{figure}

\newpage

\section*{Теоретическая часть}

\subsection*{Классификация рекурсивных функций}

\textbf{Рекурсия} -- ссылка на описываемый объект во время его описания.

\begin{enumerate}
    \item Простая - рекурсивный вызов встречается один раз.
    \item Второго порядка - несколько рекурсивных вызовов.
    \item Взаимная - несколько функций, которые могут друг друга взаимно вызывать.
    \item Хвостовая рекурсия - результат строится на входе в рекурсию.
    На выходе ничего считать не нужно.
    
    Для преобразования не хвостовой рекурсии в хвостовую можно использовать
    дополнительные параметры, в которые при каждом вызове рекурсии будет 
    формироваться и записываться результат.
    В этом случае необходимо использовать функцию-оболочку для запуска рекурсивной функции с начальными значениями дополнительных параметров.
 
    % рекурсивный вызов последний.
    \item Дополняемая рекурсия - использует дополнительную функцию.
    (результат рекурсии используется, как аргумент некоторой другой функции, которая называется дополняемой функцией)
\end{enumerate}

\end{document}