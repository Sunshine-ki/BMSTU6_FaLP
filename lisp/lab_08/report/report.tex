\documentclass[a4paper,oneside,12pt]{extreport}

\include{preamble}


\begin{document}

\include{title}

\section*{Практическая часть}

\begin{task}
    Напишите функцию, которая умножает на заданное число-аргумент все числа из заданного списка-аргумента, когда
	
    a) все элементы списка — числа,
    \begin{lstlisting}[language=Lisp]
(defun f (lst num)
    (cond ((null lst) ())
    (T (cons (* num (car lst)) (f (cdr lst) num))) ) )
\end{lstlisting}

    Пример использования:
    \begin{lstlisting}[language=Lisp]    
(f '(1 2 3 4) 5) ;; (5 10 15 20)
    \end{lstlisting}

    б) элементы списка — любые объекты.
    \begin{lstlisting}[language=Lisp]
(defun f (lst num)
    (cond ((null lst) ())
    ((symbolp (car lst)) (cons (car lst) (f (cdr lst) num)))
    ((listp (car lst)) (cons (f (car lst) num) (f (cdr lst) num)))
    (T (cons (* num (car lst)) (f (cdr lst) num))) ) )
    \end{lstlisting}

    Пример использования:
    \begin{lstlisting}[language=Lisp]    
(f '(1 2 (3 4 a) (b) T 7) 2) ;; (2 4 (6 8 A) (B) T 14)
    \end{lstlisting}
\end{task}

\begin{task}
    Напишите функцию, \code{select-between}, которая из списка-аргумента,
	содержащего только числа, выбирает только те, которые расположены между
    двумя указанными границами-аргументами и возвращает их в виде списка

    \begin{lstlisting}[language=Lisp]
(defun check-border (x a b)
    (and (>= x a) (<= x b)) )

(defun select-between (lst a b)
	(cond ((null lst) ())
	((symbolp (car lst)) (cons (car lst) (select-between (cdr lst) a b)))
	((listp (car lst)) (cons (select-between (car lst) a b) (select-between (cdr lst) a b)))
	((check-border (car lst) a b) (cons (car lst) (select-between (cdr lst) a b)))
	(T (select-between (cdr lst) a b))) )
    \end{lstlisting}

    Пример использования:
    \begin{lstlisting}[language=Lisp]    
(select-between '(1 2 (a b 3 4) T c 4 6 11 5) 2 7) ;; (2 (A B 3 4) T C 4 6 5)
    \end{lstlisting}
\end{task}

\begin{task}
    Что будет результатом (mapcar 'вектор '(570-40-8))

    mapcar примерняет свой первый аргумент поэлементно к своим аргументам.
    Т.е. первым аргументом должна быть функция. В нашем случае функции 'вектор' нет. 

    Результат: Error: ВЕКТОР is undefined.
\end{task}


\begin{task}
    Напишите функцию, которая уменьшает на 10 все числа из списка аргумента этой функции.

    \begin{lstlisting}[language=Lisp]
(defun f-func (lst)
    (mapcar (lambda (x) (- x 10)) lst))
    \end{lstlisting}

    \begin{lstlisting}[language=Lisp]
(defun f-rec (lst)
    (cond ((null lst) ())
    ((symbolp (car lst)) (cons (car lst) (f-rec (cdr lst))))
    ((listp (car lst)) (cons (f-rec (car lst)) (f-rec (cdr lst))))
    (T (cons (- (car lst) 10) (f-rec (cdr lst))))) )    
    \end{lstlisting}

    Пример использования:
    \begin{lstlisting}[language=Lisp]  
(f-func '(11 12 13 14 1))       ;; (1 2 3 4 -9)  
(f-rec '(11 12 13 14 1))        ;; (1 2 3 4 -9)
(f-rec '(11 12 (14 b 15) 16))   ;;(1 2 (4 B 5) 6)
    \end{lstlisting}
\end{task}

\begin{task}
    Написать функцию, которая возвращает первый аргумент
    списка-аргумента, который сам является непустым списком.

    \begin{lstlisting}[language=Lisp]
(defun f (lst)
	(cond ((null lst) NIL)
	((null (car lst)) NIL)
	(T (car lst))) )
    \end{lstlisting}

    Пример использования:
    \begin{lstlisting}[language=Lisp]    
(f '(Nil 1 2 3)) ;; NIL
(f '((1 2 3) 4 5 6)) ;; (1 2 3)
    \end{lstlisting}
\end{task}
    
\begin{task}
    Сумма числовых элементов смешанного структурированного списка
    \begin{lstlisting}[language=Lisp] 
(defun f-rec (lst num)
	(cond ((null lst) num)
	((symbolp (car lst)) (f-rec (cdr lst) num))
	((listp (car lst)) (+ (f-rec (car lst) 0) (f-rec (cdr lst) num)))
	(T (f-rec (cdr lst) (+ num (car lst))))) )

(defun f (lst)
	(f-rec lst 0) )   
    \end{lstlisting}

    Пример использования:
    \begin{lstlisting}[language=Lisp]    
(f '(1 2 3 (a b c) (a 2 b) (((c))) ((5)))) ;; => 13
    \end{lstlisting}
\end{task}

% \begin{figure}[ht!]
% 	\centering{
% 		\includegraphics[width=0.5\textwidth]{img/5.png}
% 		\caption{Результат работы 5.} }
% \end{figure}

\newpage

\section*{Теоретическая часть}

\subsection*{Порядок работы и варианты использования функционалов}

Функционалы - функции, которые в качестве параметра принимают функцию.

Применяющие функционалы позволяют применить свой функциональный аргумент (функцию) к заданным её аргументам.

Функционал \textbf{apply} является функцией с двумя вычисляемыми аргументами, обращение к ней имеет вид: 

\begin{lstlisting}[language=Lisp] 
    (apply F L)
\end{lstlisting}

где $F$ – функциональный аргумент и $L$ – список: $L =>(p_1 ... p_n )$, $n \geq 0$
рассматриваемый как список фактических параметров для $F$. Значение
функционала – результат применения $F$ к этим фактическим параметрам,

Пример:
\begin{lstlisting}[language=Lisp] 
(apply #'* '(2 5))                  ;; => 10
(apply (lambda (x) (- x 10)) '(2))  ;; => -8    
\end{lstlisting}

Функционал \textbf{funcall} является специальной функцией.
Обращение к ней:

\begin{lstlisting}[language=Lisp] 
(funcall F e1 ... en )
\end{lstlisting}

Где $n \geq 0$. Её действие аналогично apply, отличие состоит в том, что аргументы
применяемой функции F задаются не списком, а по отдельности.

Пример:

\begin{lstlisting}[language=Lisp] 
(funcall #'* '2 5)                  ;; => 10
(funcall (lambda (x) (- x 10)) 2)   ;; => -8    
\end{lstlisting}
    
В группу отображающих функционалов входят функции mapcar и maplist.
Их имена имеют префикс map (mapping – отображение), поскольку их
действие – отображение списка-аргумента в список-результат 
за счёт применения заданной функции к элементам исходных списков.

Обращение к \textbf{mapcar} имеет вид:

\begin{lstlisting}[language=Lisp] 
(mapcar F L1 L2 ... Ln)
\end{lstlisting}

Функционал mapcar последовательно применяет свой функциональный аргумент F (функцию
от одного аргумента) к элементам списков $L_i$, которые извлекаются из $L_i$
функцией-селектором car (поэтому имя car входит в имя функционала),
и возвращает список из полученных значений. 

Схематично результат может быть записан как:
\begin{lstlisting}[language=Lisp] 
((F (car L1) ... (car Ln)) (F (cadr L1) ... (cadr Ln)) (F (caddr L1) ... (caddr Ln)) ...).
\end{lstlisting}

Пример:
\begin{lstlisting}[language=Lisp] 
(mapcar #'length '((A B) (C) (D E F)))  ;; =>(2 1 3)
(mapcar #'list '(A S D F))              ;; => ((A)(S)(D)(F))
(mapcar (lambda (x) (if (< x 0) (* -1 x) x))  '(-1 2 3 -4 5)) ;; => (1 2 3 4 5)
(mapcar #'* '(1 2 3) '(4 5 6))          ;; => (4 10 18)
\end{lstlisting}


Обращение к \textbf{maplist} имеет вид:
\begin{lstlisting}[language=Lisp] 
(maplist F L1 ... Ln)
\end{lstlisting}

Функционал \textbf{maplist} последовательно применяет свой функциональный
аргумент $F$ к спискам-аргументам $L_i$ и их хвостовым частям (полученным из
$L$ путём отбрасывания первого элемента, первых двух элементов и т.д.) и
возвращает список из вычисленных значений. 

Схематично действие этого функционала можно записать как
\begin{lstlisting}[language=Lisp] 
((F L1 ... Ln) (F (cdr L1) ... (cdr Ln)) (F (cddr L1) ... (cddr Ln))...)
\end{lstlisting}

Пример:
\begin{lstlisting}[language=Lisp] 
(maplist #'list '(a b c d)) 
;; => (((A B C D)) ((B C D)) ((C D)) ((D)))

(maplist #'list '(1 2 3) '(4 5 6))
;; => (((1 2 3) (4 5 6)) ((2 3) (5 6)) ((3) (6)))
\end{lstlisting}

\begin{lstlisting}[language=Lisp] 
(find-if F L)
\end{lstlisting}

\textbf{find-if} применяет предикат $F$ к списку аргументу $L$, 
и возвращает первый элемент, для которого результат применяемого предиката отличен от $Nil$.
В случае, если все элементы не удовлетворяют предикату возвращается $Nil$:

Пример:    
\begin{lstlisting}[language=Lisp] 
(find-if #'evenp '(1 2 3 4 5))  ;; => 2
(find-if #'evenp '(1 3 5))      ;; => Nil
(find-if (lambda (x) (> x 0)) '(-5 1 2 5 -1)) ;; 1
\end{lstlisting}

\begin{lstlisting}[language=Lisp] 
(remove-if F L)
\end{lstlisting}
\textbf{remove-if} (\textbf{remove-if-not}) возвращает список 
без элементов для которых истин предикат $F$.

Пример:
\begin{lstlisting}[language=Lisp] 
(remove-if #'evenp '(1 2 3))        ;; => (1 3)
(remove-if-not #'evenp '(1 2 3))    ;; => (2)
\end{lstlisting}

\textbf{every} возвращает Nil, как только предикат (первый аргумент) для очередного аргумента списка-аргумента вернул Nil, 
иначе, если применение предиката для каждого элемента вернуло T, вернется T.

Пример:
\begin{lstlisting}[language=Lisp] 
(every #'symbolp '(a b c d))    ;; T
(every #'symbolp '(a b c 1))    ;; Nil

(every #'numberp '(1 2 3))      ;; T
(every #'numberp '(1 2 a))      ;; Nil
\end{lstlisting}

\textbf{some} возвращает истину, как только предикат (первый аргумент) для очередного аргумента списка-аргумента вернул истину, 
иначе, если применение предиката для каждого элемента вернуло Nil, вернется Nil.

\begin{lstlisting}[language=Lisp] 
(some #'numberp '(1 2 3)) ;; => T
(some #'numberp '(1 2 a)) ;; => T
(some #'numberp '(a b c)) ;; => Nil
\end{lstlisting}

\textbf{reduce} реализует редукцию заданного списка.
Применяет функцию $F$ к первому элементу и начальному значению $A$, 
далее результат служит аргументом для применения этой же функции и второго аргумента и т.д..

Схематично можно представить так:
\begin{lstlisting}[language=Lisp] 
    (reduce F L :initial-value A) = (F(...(F(F A e 1 ) e 2 ))...e n )
\end{lstlisting}

Пример:
\begin{lstlisting}[language=Lisp] 
(reduce #'* '(1 2 3))                                       ;; => 6
(reduce (lambda (x y) (+ x y)) '(1 2 3))                    ;; => 6
(reduce (lambda (x y) (+ x y)) '(1 2 3) :initial-value 10)  ;; => 16
\end{lstlisting}

\end{document}