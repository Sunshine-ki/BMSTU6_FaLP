\usepackage{mmap}
\usepackage[T2A]{fontenc}
\usepackage[utf8]{inputenc}
\usepackage[english,russian]{babel}


% Текст отчёта следует печатать, соблюдая следующие размеры полей:
% левое — 30 мм, правое — 15 мм, верхнее и нижнее — 20 мм.
\usepackage[left=20mm, right=15mm, top=15mm, bottom=15mm]{geometry}

% \setlength{\parindent}{1.25cm} % Абзацный отступ

\usepackage{setspace}
%\onehalfspacing % Полуторный интервал

\frenchspacing % Равномерные пробелы
\usepackage{indentfirst} % Красная строка

\usepackage{microtype}
\sloppy

\usepackage{titlesec}
\titlespacing*{\chapter}{0pt}{-30pt}{8pt}
\titlespacing*{\section}{\parindent}{*4}{*4}
\titlespacing*{\subsection}{\parindent}{*4}{*4}
\titleformat{\chapter}{\LARGE\bfseries}{\thechapter}{20pt}{\LARGE\bfseries}
\titleformat{\section}{\Large\bfseries}{\thesection}{40pt}{\Large\bfseries}

\usepackage{graphicx}
\usepackage{caption}

\usepackage[unicode,pdftex]{hyperref}
\hypersetup{hidelinks}

%% title begin
\usepackage{wrapfig}

\makeatletter
	\def\vhrulefill#1{\leavevmode\leaders\hrule\@height#1\hfill \kern\z@}
\makeatother
%% title end

%% begin code
\usepackage{listings}
\usepackage{xcolor}

\lstset{
	basicstyle=\footnotesize\ttfamily,
	breakatwhitespace=true,
	breaklines=true,
	commentstyle=\color{gray},
	frame=single,
	keywordstyle=\color{blue},
	stringstyle=\color{red},
	tabsize=8
}

\lstdefinestyle{lispinline}{
	frame=none,
	language=Lisp
}

\newcommand{\code}[1]{\texttt{#1}}
%% end code

%% begin theorem
\usepackage{amsthm}

\makeatletter
\newtheoremstyle{indented}
	{}% measure of space to leave above the theorem
	{}% measure of space to leave below the theorem
	{}% name of font to use in the body of the theorem
	{\parindent}% measure of space to indent
	{\bfseries}% name of head font
	{.}% punctuation between head and body
	{ }% space after theorem head; " " = normal interword space
	{}% header specification (empty for default)
\makeatother

\theoremstyle{indented}

\newtheorem{definition}{Определение}[section]
\newtheorem{example}{Пример}[section]
\newtheorem{theorem}{Теорема}[section]
\newtheorem{task}{Задание}

\makeatletter
\DeclareRobustCommand\bfseriesitshape{%
	\not@math@alphabet\itshapebfseries\relax
	\fontseries\bfdefault
	\fontshape\itdefault
	\selectfont
}
\makeatother

\DeclareTextFontCommand{\textbfit}{\bfseriesitshape}
\DeclareTextFontCommand{\define}{\bfseriesitshape}
%% end theorem

%% begin columns
\usepackage{etoolbox,refcount}
\usepackage{multicol}

\newcounter{countitems}
\newcounter{nextitemizecount}
\newcommand{\setupcountitems}{%
	\stepcounter{nextitemizecount}%
	\setcounter{countitems}{0}%
	\preto\item{\stepcounter{countitems}}%
}
\makeatletter
\newcommand{\computecountitems}{%
	\edef\@currentlabel{\number\c@countitems}%
	\label{countitems@\number\numexpr\value{nextitemizecount}-1\relax}%
}
\newcommand{\nextitemizecount}{%
	\getrefnumber{countitems@\number\c@nextitemizecount}%
}
\newcommand{\previtemizecount}{%
	\getrefnumber{countitems@\number\numexpr\value{nextitemizecount}-1\relax}%
}
\makeatother
\newenvironment{AutoMultiColItemize}{%
	\ifnumcomp{\nextitemizecount}{>}{3}{\begin{multicols}{2}}{}%
		\setupcountitems\begin{itemize}}%
		{\end{itemize}%
		\unskip\computecountitems\ifnumcomp{\previtemizecount}{>}{3}{\end{multicols}}{}}
\makeatother
\newenvironment{AutoMultiColEnumerate}{%
	\ifnumcomp{\nextitemizecount}{>}{3}{\begin{multicols}{2}}{}%
		\setupcountitems\begin{enumerate}}%
		{\end{enumerate}%
		\unskip\computecountitems\ifnumcomp{\previtemizecount}{>}{3}{\end{multicols}}{}}
%% end columns
