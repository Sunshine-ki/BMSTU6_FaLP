\documentclass[a4paper,oneside,12pt]{extreport}

\include{preamble}


\begin{document}

\include{title}

\section*{Практическая часть}

\begin{task}
    Разработать программу, позволяющую:
    \begin{enumerate}
        \item сформировать список из элементов числового списка, больших заданного значения;
        \item сформировать список из элементов, стоящих на нечетных позициях исходного списка (нумерация от 0);
        \item удалить заданный элемент из списка (один или все вхождения);
        \item преобразовать список в множество (можно использовать ранее разработанные процедуры).
    \end{enumerate}

    Список из элементов числового списка, больших заданного значения:
    \begin{lstlisting}[language=Prolog]
DOMAINS
    list = integer*.

PREDICATES
    f(list, integer, list).

CLAUSES
    f([H|T], Elem, [H|Res]) :-
        H > Elem, !,
        f(T, Elem, Res).
        
    f([_|T], Elem, Res) :-
        f(T, Elem, Res), !.
    
    f([], _, []) :- !.

GOAL
    f([4, 5,1, 4, 6], 3, Result).
    \end{lstlisting}

    \begin{figure}[ht!]
        \centering{
            \includegraphics[width=0.8\textwidth]{img/1.jpg}
            \caption{Результат работы первого задания}}
        \end{figure}

    \newpage

    Список из элементов, стоящих на нечетных позициях исходного списка (нумерация от 0):
    \begin{lstlisting}[language=Prolog]
DOMAINS
    list = integer*.
    
    PREDICATES
    odd(list, list).
    
CLAUSES
    odd([_,H|T], [H|Res]) :- odd(T, Res).
    odd([_], []) :- !.
    odd([],[]) :- !.
    
GOAL
    odd([0, 1, 2, 3, 4, 5, 7], Result).
    \end{lstlisting}

    \begin{figure}[ht!]
    \centering{
        \includegraphics[width=0.8\textwidth]{img/2.jpg}
        \caption{Результат работы второго задания}}
    \end{figure}


    Удаление заданного элемента из списка и преобразование списка в множество:
    \begin{lstlisting}[language=Prolog]
DOMAINS
    list = integer*.

PREDICATES
    del(integer, list, list).
    createSet(list, list).

CLAUSES
    del(Elem, [H|T], [H|Res]) :- 
        H <> Elem, !,
        del(Elem, T, Res).
        
    del(Elem, [_|T], Res) :-
        del(Elem, T, Res), !.
    
    del(_, [], []) :- !.
    
    
    createSet([H|T], [H|Res]) :-
        del(H, T, Tmp),
        createSet(Tmp, Res).
    createSet([], []).

GOAL
    del(3, [4, 3, 1, 2, 3], Result).
    % createSet([1, 2, 3, 4, 5, 6, 1, 2, 3, 4, 5, 3, 2, 6], Result).
    \end{lstlisting}
        
    \begin{figure}[ht!]
    \centering{
        \includegraphics[width=0.8\textwidth]{img/3.jpg}
        \caption{Результат работы третьего задания}}
    \end{figure}

    \begin{figure}[ht!]
    \centering{
        \includegraphics[width=0.8\textwidth]{img/4.jpg}
        \caption{Результат работы четвертого задания}}
    \end{figure}

\end{task}

\newpage


\end{document}