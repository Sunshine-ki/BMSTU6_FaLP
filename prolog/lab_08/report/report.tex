\documentclass[a4paper,oneside,12pt]{extreport}

\include{preamble}


\begin{document}

\include{title}

\section*{Практическая часть л.р.16}

\begin{task}
    Используя хвостовую рекурсию, разработать программу, позволяющую найти:

    \begin{enumerate}
        \item факториал числа;
        \item n-ое число Фибоначчи.
    \end{enumerate}

    \begin{lstlisting}[language=Prolog]
DOMAINS 
	number = integer

PREDICATES
	factorial(number, number, number).
	factorialWrapper(number, number).
	
CLAUSES
	factorial(0, Res, Res) :- !.
	factorial(Number, Current, Res) :- 
		NextNumber = Number - 1,
		Mult = Current * Number,
		factorial(NextNumber, Mult, Res).
	
	factorialWrapper(Number, -1) :- Number < 0, !. % Error. 
	factorialWrapper(Number, Res) :- factorial(Number, 1, Res).
	
GOAL
	% factorialWrapper(5, Result).
	% factorialWrapper(-5, Result).	
	factorialWrapper(0, Result).
\end{lstlisting}

    \newpage

    \begin{figure}[ht!]
    \centering{
        \includegraphics[width=0.9\textwidth]{img/res_lab_18/1.jpg}
        \caption{Факториал числа}}
    \end{figure}

    \begin{figure}[ht!]
        \centering{
            \includegraphics[width=0.9\textwidth]{img/res_lab_18/2.jpg}
            \caption{Факториал числа}}
    \end{figure}

    \begin{figure}[ht!]
    \centering{
        \includegraphics[width=0.9\textwidth]{img/res_lab_18/3.jpg}
        \caption{Факториал числа}}
    \end{figure}
    
    \begin{figure}[ht!]
        \centering{
            \includegraphics[width=0.9\textwidth]{img/res_lab_18/4.jpg}
            \caption{Факториал числа}}
    \end{figure}

    \newpage

    \begin{lstlisting}[language=Prolog]
DOMAINS 
	number = integer

PREDICATES
	fibonacci(number, number, number, number).
	fibonacciWrapper(number, number).

CLAUSES
	fibonacci(1, Prev, _, Prev) :- !.
	fibonacci(Number, Prev, Current, Res) :-
		NewNumber = Number - 1,
		Newprev = Current,
		NewCurrent = Prev + Current,
		fibonacci(NewNumber, NewPrev, NewCurrent, Res).
	
	
	fibonacciWrapper(Number, -1) :- Number < 1, !. % Error.
	fibonacciWrapper(Number, Res) :- fibonacci(Number, 1, 1, Res).
	
	
GOAL
	% fibonacciWrapper(-15, Result).
	% fibonacciWrapper(1, Result).
	% fibonacciWrapper(2, Result).
	% fibonacciWrapper(3, Result).
	fibonacciWrapper(8, Result).
    \end{lstlisting}

    \begin{figure}[ht!]
    \centering{
        \includegraphics[width=0.9\textwidth]{img/res_lab_18/5.jpg}
        \caption{Фибоначчи}}
    \end{figure}
    
    \begin{figure}[ht!]
        \centering{
            \includegraphics[width=0.9\textwidth]{img/res_lab_18/6.jpg}
            \caption{Фибоначчи}}
    \end{figure}

    \begin{figure}[ht!]
        \centering{
            \includegraphics[width=0.9\textwidth]{img/res_lab_18/7.jpg}
            \caption{Фибоначчи}}
    \end{figure}

\end{task}

\newpage

\section*{Практическая часть л.р.17}

\begin{task}
    Используя хвостовую рекурсию, разработать эффективную программу, позволяющую:

    \begin{enumerate}
        \item найти длину списка (по верхнему уровню);
        \item найти сумму элементов числового списка;
        \item найти сумму элементов числового списка, стоящих на нечетных позициях
        исходного списка (нумерация от 0).
    \end{enumerate}

    Убедиться в правильности результатов.

    Длина списка.
    \begin{lstlisting}[language=Prolog]
DOMAINS
    list = integer*.

PREDICATES
    length(list, integer, integer).
    lengthWapper(list, integer).
    
CLAUSES
    length([], Res, Res) :- !.
    length([_|T], CurrValue, Res) :- 
        NewCurrValue = CurrValue + 1,
        length(T, NewCurrValue, Res). 
    
    lengthWapper(L, Res) :- length(L, 0, Res).

GOAL
    % lengthWapper([1, 2, 3], Result).
    % lengthWapper([], Result).
    lengthWapper([1, 2, 3, 1, 2, 3, 1, 2, 3], Result).
    \end{lstlisting}

    Сумма элементов числового списка.
    \begin{lstlisting}[language=Prolog]
DOMAINS
    list = integer*.

PREDICATES
    sum(list, integer, integer).
    sumWapper(list, integer).
    
CLAUSES
    sum([], Res, Res) :- !.
    sum([H|T], CurrValue, Res) :- 
        NewCurrValue = CurrValue + H,
        sum(T, NewCurrValue, Res). 
    
    sumWapper(L, Res) :- sum(L, 0, Res).

GOAL
    % sumWapper([1, 2, 3], Result).
    % sumWapper([], Result).
    sumWapper([1, 2, 3, 1, 2, 3, 1, 2, 3], Result).
    \end{lstlisting}

    \newpage
    
    Сумма элементов числового списка, стоящих на нечетных позициях.
    \begin{lstlisting}[language=Prolog]
DOMAINS
    list = integer*.
    flag = integer. 

PREDICATES
    sum_odd(list, integer, integer, flag).
    sum_oddWapper(list, integer).
    
CLAUSES
    sum_odd([], Res, Res, _) :- !.
    sum_odd([_|T], CurrValue, Res, 0) :- sum_odd(T, CurrValue, Res, 1).
    
    sum_odd([H|T], CurrValue, Res, 1) :- 
        NewCurrValue = CurrValue + H,
        sum_odd(T, NewCurrValue, Res, 0).
    
    sum_oddWapper(L, Res) :- sum_odd(L, 0, Res, 0). 

GOAL
    % sum_oddWapper([1, 2, 3], Result).
    %sum_oddWapper([], Result).
    %sum_oddWapper([1, 2, 3, 1, 2, 3, 1, 2, 3], Result).
    sum_oddWapper([1, 2, 3, 4, 5], Result).
    %sum_oddWapper([1], Result).
    \end{lstlisting}

    \begin{figure}[ht!]
    \centering{
        \includegraphics[width=0.9\textwidth]{img/res_lab_19/1.jpg}
        \caption{Сумма элементов числового списка, стоящих на нечетных позициях.}}
    \end{figure}
    
    \begin{figure}[ht!]
        \centering{
            \includegraphics[width=0.9\textwidth]{img/res_lab_19/2.jpg}
            \caption{Сумма элементов числового списка, стоящих на нечетных позициях.}}
    \end{figure}

    \begin{figure}[ht!]
        \centering{
            \includegraphics[width=0.9\textwidth]{img/res_lab_19/3.jpg}
            \caption{Сумма элементов числового списка, стоящих на нечетных позициях.}}
    \end{figure}
\end{task}

\newpage


\end{document}