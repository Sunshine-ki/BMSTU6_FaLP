\documentclass[a4paper,oneside,12pt]{extreport}

\include{preamble}


\begin{document}

\include{title}

\section*{Практическая часть л.р.16}

\begin{task}
    Создать базу знаний: «ПРЕДКИ», позволяющую наиболее эффективным способом
    (за меньшее количество шагов, что обеспечивается меньшим количеством предложений БЗ -
    правил), используя разные варианты (примеры) одного вопроса, определить (указать: какой
    вопрос для какого варианта):

    \begin{enumerate}
        \item по имени субъекта определить всех его бабушек (предки 2-го колена);
        \item по имени субъекта определить всех его дедушек (предки 2-го колена);
        \item по имени субъекта определить всех его бабушек и дедушек (предки 2-го колена);
        \item по имени субъекта определить его бабушку по материнской линии (предки 2-го колена);
        \item по имени субъекта определить его бабушку и дедушку по материнской линии (предки 2-го колена).
    \end{enumerate}

    \begin{lstlisting}[language=Prolog]
DOMAINS 
    name = symbol.
    sex = symbol.
    
PREDICATES
    parent(name, name, sex).
    grand(name, name, sex, sex).
    
CLAUSES
    parent("Kira", "Ila", "w").
    parent("Kira", "Vitya", "m").
    
    parent("Vitya", "Elena", "w").
    parent("Vitya", "Mike", "m").
    
    parent("Ila", "Olya", "w").
    parent("Ila", "Tim", "m").
    
    grand(Child, NameGrandmother, Line, Sex) :- 
            parent(Child, NameParent, Line), 
            parent(NameParent, NameGrandmother, Sex).
        
GOAL
    % Grandmothers (1)
    % grand("Kira", NameGrandmotherR, _, "w").
    
    % Grandfathers (2)
    % grand("Kira", NameGrandfatheR, _, "m").
    
    % Grandmothers and grandfathers (3)
    % grand("Kira", NameGrand, _, _).
    
    % Maternal grandmother (4)
    % grand("Kira", NameGrandmotherR, "w", "w").
    
    % Maternal grandparents (5)
    grand("Kira", NameGrandmotherR, "w", _).
    \end{lstlisting}

    \newpage

    \begin{figure}[ht!]
    \centering{
        \includegraphics[width=0.9\textwidth]{img/res_lab_16/1.jpg}
        \caption{Поиск всех бабушек}}
    \end{figure}
    
    \begin{figure}[ht!]
        \centering{
            \includegraphics[width=0.9\textwidth]{img/res_lab_16/2.jpg}
            \caption{Поиск дедушек}}
    \end{figure}
    
    \newpage

    \begin{figure}[ht!]
        \centering{
            \includegraphics[width=0.9\textwidth]{img/res_lab_16/3.jpg}
            \caption{Поиск бабушек и дедушек}}
    \end{figure}
    
    \begin{figure}[ht!]
        \centering{
            \includegraphics[width=0.9\textwidth]{img/res_lab_16/4.jpg}
            \caption{Поиск бабушек по маминой линии}}
    \end{figure}
    
    \begin{figure}[ht!]
        \centering{
            \includegraphics[width=0.9\textwidth]{img/res_lab_16/5.jpg}
            \caption{Поиск бабушек и дедушек по материнской линии}}
    \end{figure}
\end{task}

\newpage

\section*{Практическая часть л.р.17}

\begin{task}
    В одной программе написать правила, позволяющие найти

    \begin{enumerate}
        \item Максимум из двух чисел
        \begin{enumerate}
            \item без использования отсечения,
            \item с использованием отсечения;
        \end{enumerate}
        \item Максимум из трех чисел
        \begin{enumerate}
            \item без использования отсечения,
            \item с использованием отсечения;
        \end{enumerate}
    \end{enumerate}

    \begin{lstlisting}[language=Prolog]
DOMAINS 
    number = integer

PREDICATES
    maximumTwo(number, number, number).
    maximumTwo2(number, number, number).

    maximumThree(number, number, number, number).
    maximumThree2(number, number, number, number).
    
CLAUSES
    maximumTwo(A, B, A) :- A >= B.
    maximumTwo(A, B, B) :- A < B.
    
    maximumTwo2(A, B, A) :- A >= B, !.
    maximumTwo2(_, B, B).
    
    maximumThree(A, B, C, A) :- A >= B, A >= C.
    maximumThree(_, B, C, Res) :- maximumTwo(B, C, Res).
    
    maximumThree2(A, B, C, A) :- A >= B, A >= C, !.
    maximumThree2(_, B, C, Res) :- maximumTwo(B, C, Res). 
    
    
GOAL
    maximumThree(3, 3, 3, Result).       
    \end{lstlisting}

    \begin{figure}[ht!]
    \centering{
        \includegraphics[width=0.9\textwidth]{img/res_lab_17/1.jpg}
        \caption{Максимум из двух чисел}}
    \end{figure}
    
    \begin{figure}[ht!]
        \centering{
            \includegraphics[width=0.9\textwidth]{img/res_lab_17/2.jpg}
            \caption{Максимум из двух чисел}}
    \end{figure}
    
    \begin{figure}[ht!]
        \centering{
            \includegraphics[width=0.9\textwidth]{img/res_lab_17/3.jpg}
            \caption{Максимум из двух чисел}}
    \end{figure}
    
    %  с использованием отсечения
    \begin{figure}[ht!]
        \centering{
            \includegraphics[width=0.9\textwidth]{img/res_lab_17/4.jpg}
            \caption{Максимум из двух чисел с использованием отсечения}}
    \end{figure}
    
    \begin{figure}[ht!]
        \centering{
            \includegraphics[width=0.9\textwidth]{img/res_lab_17/5.jpg}
            \caption{Максимум из двух чисел с использованием отсечения}}
    \end{figure}

    \begin{figure}[ht!]
        \centering{
            \includegraphics[width=0.9\textwidth]{img/res_lab_17/6.jpg}
            \caption{Максимум из двух чисел с использованием отсечения}}
    \end{figure}

    % Three
    \begin{figure}[ht!]
        \centering{
            \includegraphics[width=0.9\textwidth]{img/res_lab_17/three/1.jpg}
            \caption{Максимум из трех чисел}}
    \end{figure}
    
    \begin{figure}[ht!]
        \centering{
            \includegraphics[width=0.9\textwidth]{img/res_lab_17/three/2.jpg}
            \caption{Максимум из трех чисел}}
    \end{figure}
    
    \begin{figure}[ht!]
        \centering{
            \includegraphics[width=0.9\textwidth]{img/res_lab_17/three/3.jpg}
            \caption{Максимум из трех чисел}}
    \end{figure}

    \begin{figure}[ht!]
        \centering{
            \includegraphics[width=0.9\textwidth]{img/res_lab_17/three/4.jpg}
            \caption{Максимум из трех чисел}}
    \end{figure}


    %  с использованием отсечения
    \begin{figure}[ht!]
        \centering{
            \includegraphics[width=0.9\textwidth]{img/res_lab_17/three/5.jpg}
            \caption{Максимум из трех чисел с использованием отсечения}}
    \end{figure}
    
    \begin{figure}[ht!]
        \centering{
            \includegraphics[width=0.9\textwidth]{img/res_lab_17/three/6.jpg}
            \caption{Максимум из трех чисел с использованием отсечения}}
    \end{figure}
    
    \begin{figure}[ht!]
        \centering{
            \includegraphics[width=0.9\textwidth]{img/res_lab_17/three/7.jpg}
            \caption{Максимум из трех чисел с использованием отсечения}}
    \end{figure}

    \begin{figure}[ht!]
        \centering{
            \includegraphics[width=0.9\textwidth]{img/res_lab_17/three/8.jpg}
            \caption{Максимум из трех чисел с использованием отсечения}}
    \end{figure}

    \begin{figure}[ht!]
        \centering{
            \includegraphics[width=0.9\textwidth]{img/res_lab_17/three/9.jpg}
            \caption{Максимум из трех чисел с использованием отсечения}}
    \end{figure}

    \begin{figure}[ht!]
        \centering{
            \includegraphics[width=0.9\textwidth]{img/res_lab_17/three/10.jpg}
            \caption{Максимум из трех чисел с использованием отсечения}}
    \end{figure}
\end{task}

\newpage


\end{document}